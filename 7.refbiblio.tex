\documentclass[a4paper,12pt,oneside,openany]{book}	
\usepackage{layout}
\setlength{\textwidth}{15.0 cm}
\setlength{\textheight}{25.0 cm}


\usepackage[english,brazil]{babel}
\usepackage{pagina}	% pagina-padrao
\usepackage{indentfirst}		% for indent
\usepackage[latin1]{inputenc}

\begin{document}
 
\section{First section}
 
This document is an example of \texttt{thebibliography} environment using 
in bibliography management. Three items are cited: \textit{The \LaTeX\ Companion} 
book \cite{latexcompanion}, the Einstein journal paper \cite{einstein}, and the 
Donald Knuth's website \cite{knuthwebsite}. The \LaTeX\ related items are
\cite{latexcompanion,knuthwebsite}. 
 
\medskip
 
\begin{thebibliography}{9}
\bibitem{latexcompanion} 
Michel Goossens, Frank Mittelbach, and Alexander Samarin. 
\textit{The \LaTeX\ Companion}. 
Addison-Wesley, Reading, Massachusetts, 1993.
 
\bibitem{einstein} 
Albert Einstein. 
\textit{Zur Elektrodynamik bewegter K{\"o}rper}. (German) 
[\textit{On the electrodynamics of moving bodies}]. 
Annalen der Physik, 322(10):891–921, 1905.
 
\bibitem{knuthwebsite} 
Knuth: Computers and Typesetting,
\\\texttt{http://www-cs-faculty.stanford.edu/\~{}uno/abcde.html}
\end{thebibliography}
 

% ---------------------------------------------------------------
\normalsize
\cleardoublepage
\addcontentsline{toc}{chapter}{Bibliografia}
\bibliographystyle{coppe}
\bibliography{6.biblio}

\end{document}

% Copyright
      \vspace{0.5cm}

UNIVERSIDADE FEDERAL DO RIO DE JANEIRO \\
Escola Politécnica - Departamento de Eletrônica e de Computação \\
Centro de Tecnologia, bloco H, sala H-217, Cidade Universitária \\ 
Rio de Janeiro - RJ      CEP 21949-900\\
\vspace{0.5cm}
\paragraph{}Este exemplar é de propriedade da Universidade Federal do Rio de Janeiro, que poderá incluí-lo em base de dados, armazenar em computador, microfilmar ou adotar qualquer forma de arquivamento.
\paragraph{}É permitida a menção, reprodução parcial ou integral e a transmissão entre bibliotecas deste trabalho, sem modificação de seu texto, em qualquer meio que esteja ou venha a ser fixado, para pesquisa acadêmica, comentários e citações, desde que sem finalidade comercial e que seja feita a referência bibliográfica completa.
\paragraph{}Os conceitos expressos neste trabalho são de responsabilidade do(s) autor(es).


\pagebreak

% Dedicatória
\begin{center}
\textbf{DEDICATÓRIA}
\end{center}
      \vspace*{\fill}

{\raggedleft\vfill\itshape{%

Dedico este trabalho à minha família,\\ aos meus amigos e à minha namorada\\por todo apoio nesta etapa da minha vida.
}\par
}

\pagebreak


% Agradecimento
\begin{center}
\textbf{AGRADECIMENTO}
\end{center}
      \vspace{0.5cm}

Agradeço à minha orientadora, Mariane Rembold Petraglia, por todo o apoio, dedicação e paciência para a realização deste trabalho. 

Aos meus amigos, que sempre estiveram presentes durante toda a graduação.

À minha família por todo o incentivo que recebi ao longo da vida.

\pagebreak


% Resumo
\begin{center}
\textbf{RESUMO}
\end{center}
      \vspace{0.5cm}

\paragraph{}
Este trabalho dedica-se ao estudo comparativo entre algoritmos propostos para o problema de separação cega de fontes no domínio da frequência para sinais de voz em ambientes reverberantes. Inicialmente, apresenta-se o problema direcionado para misturas instantâneas e sua relação com o princípio da análise de componentes independentes (ICA). Em um segundo momento, introduz-se o conceito de misturas convolutivas lineares e suas diferenças em relação ao caso anterior, além dos métodos propostos pela literatura a para a solução do problema. Por fim, dois destes métodos são escolhidos com a finalidade de comparar o desempenho entre os mesmos.

\paragraph{}
\noindent Palavras-Chave: fdbss, ica, verossimilhança, jade, natural-ica.

\pagebreak


% Abstract
\begin{center}
\textbf{ABSTRACT}
\end{center}
      \vspace{0.5cm}

\paragraph{}
This work is dedicated to the comparative study between algorithms proposed for the problem of frequency domain blind source separation for voice signals in reverberant environments. Initially, we present the problem directed to instantaneous mixtures and its relation with the principle of Independent Component Analysis (ICA). In a second moment, we introduce the concept of linear convolution mixtures and their differences in relation to the previous case, in addition to the methods proposed by the literature to solve the problem. Finally, two methods are chosen for a purpose of comparison or performance between them.

\paragraph{}
\noindent Key-words: fdbss, ica, jade, natural-ica.

\pagebreak


% Siglas
\begin{center}
\textbf{SIGLAS}
\end{center}
      \vspace{0.5cm}

\paragraph{}BSS  - \textit{Blind Source Separation}
\paragraph{}DFT  - \textit{Discrete Fourier Transform}
\paragraph{}FFT  - \textit{Fast Fourier Transform}
\paragraph{}IDFT  - \textit{Inverse Discrete Fourier Transform}
\paragraph{}STFT - \textit{Short Time Fourier Transform}
\paragraph{}ICA  - \textit{Independent Component Analysis}
\paragraph{}FIR  - \textit{Finite Impulse Response}
\paragraph{}SIR  - \textit{Signal-to-Interference Ratio}
\paragraph{}SNR  - \textit{Signal-to-Noise Ratio}
\paragraph{}SAR  - \textit{Signal-to-Artifact Ratio}
\paragraph{}HOS  - \textit{High Order Statistics}
\paragraph{}p.d.f  - \textit{Probability Density Function}
\paragraph{}JADE  - \textit{Joint Approximate Diagonalization of Eigenmatrices}
\pagebreak








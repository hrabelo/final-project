% Copyright
      \vspace{0.5cm}

UNIVERSIDADE FEDERAL DO RIO DE JANEIRO \\
Escola Politécnica - Departamento de Eletrônica e de Computação \\
Centro de Tecnologia, bloco H, sala H-217, Cidade Universitária \\ 
Rio de Janeiro - RJ      CEP 21949-900\\
\vspace{0.5cm}
\paragraph{}Este exemplar é de propriedade da Universidade Federal do Rio de Janeiro, que poderá incluí-lo em base de dados, armazenar em computador, microfilmar ou adotar qualquer forma de arquivamento.
\paragraph{}É permitida a menção, reprodução parcial ou integral e a transmissão entre bibliotecas deste trabalho, sem modificação de seu texto, em qualquer meio que esteja ou venha a ser fixado, para pesquisa acadêmica, comentários e citações, desde que sem finalidade comercial e que seja feita a referência bibliográfica completa.
\paragraph{}Os conceitos expressos neste trabalho são de responsabilidade do(s) autor(es).


\pagebreak

% Dedicatória
\begin{center}
\textbf{DEDICATÓRIA}
\end{center}
      \vspace*{\fill}

{\raggedleft\vfill\itshape{%

Dedico este trabalho à minha família,\\ aos meus amigos e à minha namorada\\por todo apoio nesta etapa da minha vida.
}\par
}

\pagebreak


% Agradecimento
\begin{center}
\textbf{AGRADECIMENTO}
\end{center}
      \vspace{0.5cm}

Meus mais profundos e sinceros agradecimentos:
\begin{itemize}
    \item À minha orientadora, Mariane Rembold Petraglia, por todo o apoio, dedicação e paciência para a realização deste trabalho;
    \item Aos meus amigos da graduação, com quem sempre pude contar durante este período;
    \item À minha família e minha namorada, por todo o incentivo que recebi para a conclusão da minha graduação.
\end{itemize}

\pagebreak


% Resumo
\begin{center}
\textbf{RESUMO}
\end{center}
      \vspace{0.5cm}

\paragraph{}
Este trabalho dedica-se ao estudo comparativo entre algoritmos propostos para o problema de separação cega de fontes (BSS) no domínio da frequência para sinais de voz em ambientes reverberantes. Inicialmente, apresenta-se o problema de BSS e sua relação com o princípio da análise de componentes independentes (ICA). Em um segundo momento, utilizamos estes conceitos no domínio da frequência e apresentamos os métodos propostos pela literatura para a solução do problema. Por fim, dois destes métodos são escolhidos com a finalidade de comparar seus desempenhos.

\paragraph{}
\noindent Palavras-Chave: FDBSS, ICA, JADE, NATURAL-ICA, ICA-EBM.

\pagebreak


% Abstract
\begin{center}
\textbf{ABSTRACT}
\end{center}
      \vspace{0.5cm}

\paragraph{}

This work is dedicated to the comparative study between algorithms proposed for solving the problem of blind source separation (BSS) in the frequency domain for voice signals in reverberant environments. Initially, we present the problem of BSS and its relationship with the principle of independent component analysis (ICA). In a second moment, we use these concepts in the frequency domain and present the methods proposed in the literature for obtaing the BSS solution. Finally, two of these methods are chosen for the purpose of comparing their performances.

\paragraph{}
\noindent Key-words: FDBSS, ICA, JADE, NATURAL-ICA, ICA-EBM.

\pagebreak


% Siglas
\begin{center}
\textbf{SIGLAS}
\end{center}
      \vspace{0.5cm}

\paragraph{}BSS  - \textit{Blind Source Separation}
\paragraph{}DFT  - \textit{Discrete Fourier Transform}
\paragraph{}FFT  - \textit{Fast Fourier Transform}
\paragraph{}IDFT  - \textit{Inverse Discrete Fourier Transform}
\paragraph{}STFT - \textit{Short Time Fourier Transform}
\paragraph{}ICA  - \textit{Independent Component Analysis}
\paragraph{}PCA  - \textit{Principal Component Analysis}
\paragraph{}FIR  - \textit{Finite Impulse Response}
\paragraph{}SIR  - \textit{Signal-to-Interference Ratio}
\paragraph{}SNR  - \textit{Signal-to-Noise Ratio}
\paragraph{}SAR  - \textit{Signal-to-Artifact Ratio}
\paragraph{}HOS  - \textit{High Order Statistics}
\paragraph{}p.d.f  - \textit{Probability Density Function}
\paragraph{}JADE  - \textit{Joint Approximate Diagonalization of Eigenmatrices}
\pagebreak








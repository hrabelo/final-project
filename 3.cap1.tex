\section{Proposta do trabalho}

Nós somos rodeados por sons. Quanto mais ruidoso é um ambiente, mais difícil se torna identificar um som em particular. Motivado por isto, especialistas na área de processamento de sinais desenvolveram ferramentas para separar e extrair uma informação sonora a partir de uma amostra ruidosa, tanto para comunicação humano-máquina quanto humano-humano. 

Conhecida como separação cega de fontes, do inglês \textit{blind source separation} (BSS), esta é uma abordagem que visa estimar os sinais de áudio importantes através apenas das suas misturas observadas. Assim sendo, a estimação é obtida sem possuir quaisquer informações sobre o processo de mistura ou sobre as fontes, como sua estrutura na frequência ou temporal.

A proposta teve origem na área de biomedicina, como será visto mais adiante, mas atualmente é bastante estudada na área de engenharia eletrônica. Mais especificamente, pode-se citar a área de processamento de sinais como a que há mais trabalhos no assunto, uma vez que uma das aplicações principais do BSS é a de separação de sinais de voz.

\section{Delimitação}

Este trabalho abordará o caso da separação cega de fontes para misturas convolutivas, no domínio da frequência, determinadas (quando o número de fontes é igual ao número de sensores) de sinais de voz. Desta forma, deve-se dizer que a sua aplicação é limitada a casos em que necessariamente dispõe-se da mesma quantidade de sensores e de fontes. Entretanto, é uma boa oportunidade de aplicar fundamentos de diversas áreas, tais como álgebra linear, estatística e otimização, e combiná-los de forma a obter um resultado satisfatório para o problema de BSS.


\section{Justificativa}

A interdisciplinaridade do assunto é fascinante. Combinar diversos fundamentos e aplicá-los a um problema do mundo real é uma demonstração clara porque todo conhecimento é valioso. Além disso, a diversidade de aplicações faz com que problemas de naturezas diferentes possuam uma solução similiar. Estudar e dominar as técnicas desenvolvidas para BSS é fundamental para o avanço cientifico e tecnológico da área de processamento de sinais.


\section{Objetivos}

O objetivo principal deste trabalho é comparar dois diferentes métodos empregados na separação cega de fontes para o mesmo cenário. Entretanto, para chegar a este ponto, é necessário introduzir o problema do BSS, apresentar sua modelagem matemática, descrever os fundamentos utilizados nesta formulação e propor um cenário para a realização dos testes e obtenção dos dados.


\section{Metodologia}

O assunto em questão requer um conhecimento sólido em matemática, estatística e processamento de sinais para o seu entendimento. Apesar de grande parte do problema já ser bem definido e ter algumas soluções adequadas para casos particulares, existem algumas etapas em que se encontra uma dificuldade maior, como no caso da permutação quando se trabalha no domínio da frequência (que será visto mais adiante). Assim sendo, serão revistos alguns destes conceitos antes da apresentação do problema em si. Uma vez feito isto, o problema da BSS será descrito para o caso de sinais de voz em ambientes reverberantes. Serão discutidas as dificuldades deste caso e suas respectivas propostas de solução.


\section{Descrição}

No Capítulo \ref{chap:2} apresenta-se o problema do BSS, bem como os conceitos utilizados para a modelagem do problema.

O Capítulo \ref{chap:3} introduz os métodos de separação cega de fontes no domínio da frequência, apresentando suas vantagens e desvantagens. Além disso, propõe-se soluções para os problemas gerados por esta abordagem, tal como o de permutação.

No Capítulo \ref{chap:4} são apresentados o ambiente de simulação e a análise comparativa entre os algoritmos de separação.

O Capítulo \ref{chap:5} apresenta a conclusão do trabalho e introduz um contexto para trabalhos futuros.


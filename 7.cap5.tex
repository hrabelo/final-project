\label{chap:5}
Este trabalho abordou a separação cega de fontes do domínio da frequência para ambientes reverberantes. Por ser uma abordagem que retrata melhor a realidade, escolhemos o caso de misturas convolutivas e as simulações foram feitas em um ambiente reverberante.

No Capítulo \ref{chap:2}, apresentamos uma breve descrição do problema da BSS, além da simplificação da sua modelagem matemática para o caso linear, determinado e convolutivo, chegando no modelo matricial. Vimos dois problemas relacionados à ambiguidade em BSS, que são a incapacidade de recuperar a ordem e a amplitude das fontes sem ter informações adicionais sobre as mesmas.Ainda neste capítulo, introduzimos o conceito de ICA e mostramos como este pode servir para resolver o problema do BSS. Nesta etapa, exploramos o conceito de separabilidade de um sistema e, dessa forma, mostramos que estatísticas de segunda ordem (correlação) não são suficientes para recuperar os sinais das fontes, uma vez que ainda temos uma indeterminação com relação a uma matriz de rotação (ortogonal). Entretanto, o procedimento de resolução por estatísticas de segunda ordem pode ser utilizado como uma poderosa ferramenta de pré-processamento, correspondendo ao branqueamento dos sinais, que em muitos algoritmos de separação é um pré-requisto (FastICA). 

Apresentamos dois tipos de abordagens para a ICA, a de maximização da não-gaussianidade e de estimação da máxima verossimilhança. No primeiro caso, derivamos os algoritmos que utilizam curtose e negentropia. Ainda no caso de maximização da não-gaussianidade, devido a problemas com os algoritmos que utilizam gradiente, apresentamos a abordagem de ponto fixo e o algoritmo FastICA. No segundo caso, é necessário uma estimativa da distribuição das fontes, que mesmo sendo grosseira, resulta em boas soluções. A independência da matriz de misturas $\mathbf{A}$ é o que torna esta abordagem bastante atrativa, pois converge rapidamente mesmo quando $\mathbf{A}$ é mal condicionada. A combinação do FastICA com o Natural ICA formam a base de separação no domínio da frequência. Como último tópico deste capítulo, apresentamos formas de medir o desempenho dos métodos através do SIR, SAR e SDR.

O Capítulo \ref{chap:3} apresenta o conceito da separação cega de fontes no domínio da frequência, que transforma os sinais do domínio do tempo para a frequência e, como consequência, transforma a ICA convolutiva em um problema de $\K$ ICAs instantâneas, onde ${K}$ é o número de raias da FFT. Neste caso, as ambiguidades de permutação e escalamento são um grande problema. Também foi apresentada a forma de converter os sinais do domínio do tempo para o da frequência através da STFT. É comum utilizar-se de janelas de análise na STFT, que devem atender à COLA para que não haja distorção na transformação. Determinamos o uso da janela de \textit{Hanning} com o salto $\mathpzc{J}$ = $\frac{L}{2}$ como janela de análise e da janela retangular com  $\mathpzc{J}$ = $\frac{L}{4}$ como janela de síntese. As janelas de análise podem ser aplicadas em qualquer situação, mas as janelas de síntese não fazem sentido se o objetivo for a implementação de convoluções lineares, como em BSS. Neste caso, a restrição de que o número de raias ${K}$ deve ter aproximadamente o tamanho da janela somado ao comprimento do filtro deve ser seguida para que a reconstrução seja perfeita. 

Vimos a importância do branqueamento como etapa de pré-processamento, essencial no caso do FastICA e um passo muito útil se utilizado o Natural ICA. Como o Natural ICA é baseado em gradiente, ele sofre influência da escolha do passo de adaptação. Se a potência do sinal for muito diferente de uma raia pra outra, é necessário ajustar o passo de adaptação por raia para que a convergência seja uniforme. Branqueando os sinais, esta potência é normalizada e o problema é resolvido. Vimos que o branqueamento faz aproximadamente metade do trabalho de separação por um custo computacional muito menor do que o do ICA.
 
Sobre a separação dos sinais, apresentamos primeiramente o algoritmo Natural ICA, que é resultado da estimação da máxima verossimilhança. Para fins de performance, adotamos uma abordagem em que primeiro aplicamos o FastICA e a sua matriz separadora resultante é utilizada como matriz inicial no algoritmo Natural ICA, ao invés da matriz branqueadora. Utilizamos a função \textit{score} do Natural ICA na forma polar, segundo \cite{LuizVictorio}. Em seguinda, apresentamos o algoritmo ICA-EBM, que é uma variação do FastICA e se baseia na maximização da não-gaussianidade através da negentropia. Esta variação consiste basicamente em tentar impor fronteiras à entropia $\mathpzc{H(x)}$ do vetor de misturas $\mathbf{x}$ para minimizá-la e, então, obter o vetor de estimativa das fontes $\mathbf{y}$ mais independente possível. Também apresentamos o algoritmo JADE, que utiliza das \textit{HOS} para criar a matriz e o vetor de cumulantes associados ao sistema de mistura e visa estimar os ângulos que minimizam a expressão dada por um conjunto de matrizes de rotação de forma a obter vetores independentes.

No Capítulo \ref{chap:4}, apresentamos o cenário de simulação e comparamos os algoritmos Natural ICA + FastICA e ICA-EBM através das medidas propostas no final do Capítulo \ref{chap:2}, dando ênfase na análise para diferentes tipos de reverberação ${T_s}$. Podemos ver resultados ligeiramente melhores do ICA-EBM em relação ao NaturalICA em todos os tempos de reverberação. Entretanto, com o aumento deste tempo, vemos que os resultados de ambos passa a se deteriorar. Por não ser um resultado significativamente melhor e o tempo de convergência do algoritmo ser consideravelmente maior, o ICA-EBM não se mostra vantajoso em relação à abordagem do NaturalICA + FastICA.

\section{Trabalhos Futuros}

Devido à versatilidade do assunto, é possível propor diversas alternativas para trabalhos futuros. Entretanto, atendo-se ao conjunto de abordagens escolhidas para a realização deste trabalho, podemos sugerir:
\begin{itemize}
    \item Adaptação do método para abordagem de problemas subdeterminados, ou seja, onde o número de sensores é menor do que o número de fontes. Por ser um caso mais próximo da realidade, onde nem sempre dispõe-se do mesmo número de fontes e sensores, sua aplicabilidade seria bem mais vasta;
    
    \item Estudo visando a implementação do método de separação \textit{on-line}. Com um sistema operando em tempo-real, seria possível estender sua aplicação para diversas áreas, tal como supervisionamento de ambientes, controle e reconhecimento por voz;
    
    \item Abordagem utilizando filtragem adaptativa em subbandas, que poderia melhorar o desempenho da análise em frequência e, portanto, o resultado da separação das fontes.
\end{itemize}

Este trabalho abordou a separação cega de fontes do domínio da frequência para ambientes reverberantes. Por uma abordagem que retratasse melhor a realidade, escolhemos o caso convolutivo e as simulações foram feitas em um ambiente reverberante.

No capítulo \ref{chap:2}, apresentamos um pequeno cronograma do problema da BSS, além da simplificação da sua modelagem matemática para o caso linear, determinado e convolutivo, chegando no modelo matricial. Vimos dois problemas relacionados à ambiguidade em BSS, que são a incapacidade de recuperar a ordem e a amplitude das fontes sem conhecer informações sobre as mesmas. 

Ainda neste capítulo, introduzimos o conceito de ICA e mostramos como este pode servir para resolver o problema do BSS. Nesta etapa, exploramos o conceito de separabilidade de um sistema e, dessa forma, mostramos que estatísticas de segunda ordem (correlação) não são suficientes para a separação das fontes, uma vez que ainda temos uma indeterminação com relação à uma matriz de rotação (ortogonal). Entretanto, o procedimento de resolução por estatísticas de segunda ordem pode ser utilizada como uma poderosa ferramenta de pré-processamento, chamada de branqueamento, que em muitos casos de separação, é um pré-requisto (Fast ICA). 

Apresentamos dois tipos de abordagens para a ICA, a de maximização da não-gaussianide e de estimação da máxima verossimilhança. No primeiro caso, derivamos os algoritmos que utilizam curtose e negentropia, comparando entre elas. Ainda no caso de maximização da não-gaussianidade, devido à problemas com os algoritmos que utilizam gradiente, apresentamos à abordagem de ponto fixo e o algoritmo FastICA. No segundo caso, é necessário uma estimativa da distribuição das fontes, que mesmo sendo grosseiro, resulta em boas soluções. A independência da matriz de misturas $\mathbf{A}$ é o que torna esta abordagem bastante atrativa, pois converge rapidamente mesmo quando ela é mal condicionada. A combinação do FastICA com o Natural ICA formam a base de separação no domínio da frequência. Como último item deste capítulo, apresentamos formas de medir o desempenho dos métodos através do SIR, SAR e SDR.

O capítulo \ref{chap:3} apresenta o conceito da Separação Cega de Fontes no Domínio da Frequência, que transforma os sinais do domínio do tempo para a frequência e, como consequência, transforma a ICA convolutiva num problema de $\mathpzc{K}$ ICAs instantâneas, onde $\mathpzc{K}$ é o número de raias da FFT. Neste caso, as ambiguidades de permutação e escalamento são um grande problema. Também foi apresentada a forma de converter o domínio do tempo para frequência através da STFT.

É comum utilizar-se de janelas de análise na STFT, que devem atender à COLA para que não haja distorção na transformação. Determinamos o uso da janela de \textit{Hanning} com o salto $\mathpzc{J}$ = $\frac{L}{2}$ para a janela de análise e da janela retangular com  $\mathpzc{J}$ = $\frac{L}{4}$ como janela de síntese. As janelas de análise podem ser aplicadas em qualquer situação, mas as janelas de síntese não fazem sentido se o objetivo for a implementação de convoluçõeslineares, que é o caso de BSS. Neste caso, a restrição de que o número de raias $\mathpzc{K}$ deve ter aproximadamente o tamanho da janela somado ao tamanho do filtro deve ser seguido, para que a reconstrução seja perfeita. 

Vimos a importância do branqueamento como etapa de pré-processamento, que é essencial no caso do FastICA e um passo muito útil se utilizado o Natural ICA. Como o Natural ICA é baseado em gradiente, ele sofre bastante influência do passo de adaptação.Se a potência do sinal for muito diferente de uma raia pra outra, é necessário ajustar o passo de adaptação por raia para que a convergência seja uniforme. Branqueando os sinais, esta potência é normalizada e o problema é resolvido. Vimos que o branqueamento faz aproximadamente metade do trabalho de separação por um custo computacional muito menor do que o do ICA.
 
Sobre a separação dos sinais, utilizamos primeiramente o algoritmo de Natural ICA, que é resultado da estimação da máxima verossimilhança. Para fins de performance, adotamos uma abordagem em que primeiro aplicamos o FastICA e a sua matriz separadora resultante é utilizada como matriz inicial no algoritmo Natural ICA, em vez da matriz branqueadora. Utilizamos a função \textit{score} do Natural ICA na forma polar, segundo o trabalho de \cite{LuizVictorio}. Também utilizamos o método algébrico JADE, que se baseia nos cumultantes de quarta ordem e a utilização de tensores. Este modelo consiste basicamente em tentar encontrar uma rotação ortogonal do vetor de misturas $\mathbf{x}$ para estimar os vetores de estimativa das fontes $\mathbf{y}$ de forma que o módulo de sua curtose seja máximo.

Por fim, comparamos estes dois métodos de separação através das medidas propostas no final do capítulo \ref{chap:2}, dando ênfase na análise do algoritmo Natural ICA para diferentes tipos de reverberação $\mathbf{T_s}$. Podemos ver uma performance melhor do Natural ICA em relação ao JADE para o caso em que a reverberação do ambiente é de 0.1s. Entretanto, com o aumento deste tempo, vemos que a performance deste método passa a deteriorar.

\section{Trabalhos Futuros}

Devido à versatilidade do assunto, é possível propor diversas alternativas para trabalhos futuros. Entretanto, atendo-se ao conjunto de abordagens escolhidas para a realização deste trabalho, podemos sugerir:
\begin{itemize}
    \item Adaptação do método para abordagem de problemas subdeterminados, ou seja, onde o número de sensores é menor do que o número de fontes. Por ser um caso mais próximo da realidade, onde nem sempre dispõe-se do mesmo número de fontes e sensores, sua aplicabilidade seria bem mais vasta. 
    
    \item Estudo visando a implementação do método de separação \textit{on-line}. Com um sistema operando em tempo-real, seria possível estender sua aplicação para diversas áreas, tal como supervisionamento de ambientes, controle e reconhecimento por voz e etc.
    
    \item Abordagem utilizando filtragem adaptativa em subbandas, como que poderia melhorar o desempenho da análise em frequência e, portanto, o resultado da separação das fontes.
\end{itemize}
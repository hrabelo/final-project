\label{chap:4}
\section{Simulação}\label{sec:simulation}

Simulamos a resposta em frequência de uma sala utilizando o algoritmo Image-Source Model \cite{simulation}.

A sala utilizada nas simulações tem dimensões $4,45m \times 3,55m \times 2,5m$ (largura × comprimento × altura) e é completamente selada. Além disso, o coeficiente de absorção de todas as suas paredes é o mesmo, considerando que todas são feitas do mesmo material. O material utilizado altera o tempo de reverberação, sendo que o algoritmo de simulação calcula o coeficiente de absorção das paredes da sala dependendo do tempo de reverberação escolhido. O arranjo de microfones é linear e foi posicionado em torno do ponto $Mic_c$ = [2 1,5 1,6]$^T$. Foi simulado apenas o cenário com 2 microfones e 2 fontes. As fontes foram distribuídas em torno do centro do arranjo, com dois parâmetros para identificá-las: o DOA de cada uma, e a distância delas até o arranjo. O parâmetro mais importante da sala é o tempo de reverberação. O tempo utilizado nesta dissertação é o ${T_{60}}$, que é o tempo requerido para que a energia das componentes relativas a reflexões do sinal caia a 60 dB abaixo do nível do som direto. A janela utilizada para a STFT é a janela de \textit{Hanning} de comprimento $L$ igual à $4096$ pontos.

As fontes utilizadas foram do SASSEC, correspondendo a dois trechos de sinais de voz de 10 segundos de duração cada, amostrados a 16 kHz. Um sinal é composto de uma voz masculina, enquanto o outro é composto por uma voz feminina. Decimamos os sinais para que a frequência de amostragem fosse reduzida para 8 kHz. As informações do ambiente simulado são dadas na Figura \ref{fig:environment}.

\begin{figure}
    \centering
    \includegraphics{environment.JPG}
    \caption{Configuração da sala utilizada nos testes. A sala possui as dimensões $4,45m \times 3,55m \times 2,5m$ (comprimento, largura e altura). Todo o ambiente é selado, i.e., não há portas ou janelas. Os microfones (sensores) encontram-se separados por uma distância de $4cm$ e o centro deste arranjo é representado pelo vetor $Mic_c$ = [2 1,5 1,6]$^T$. As fontes estão à distância de $1m$ do centro do arranjo de microfones e estão dispostas a $\frac{2\pi}{3}$ e $\frac{\pi}{6}$ em relação à ordenada.}
    \label{fig:environment}
\end{figure}


 \section{Análise dos Algoritmos}\label{sec:analysis}
    
    Nas condições de simulação descritas na Seção \ref{sec:simulation}, comparamos os métodos ICA-EBM e Natural ICA em relação aos resultados e à performance quanto à separação dos sinais. Vale ressaltar que o objetivo destes experimentos foi avaliar unicamente a etapa de separação dos sinais e, por isso, não houve qualquer alteração no método de resolução da permutação e de escalonamento. 
    
    Inicialmente, a simulalação foi feita com tempo de reverberação $T_{60}$ igual a 0.1s e comprimento da janela $L=2048$. Neste cenário, os sinais das fontes e dos sensores são mostrados nas Figuras \ref{fig:sources} e \ref{fig:sensors}, respectivamente. Após aplicar a STFT (definida na Seção \ref{sec:stft}) para transformar os sinais para o domínio da frequência e fazer a separação das fontes, incluindo tanto a etapa de pré-processamento (descrita na Seção \ref{sec:whitening}) quanto a de pós-processamento (apresentada na Seção \ref{sec:tdoa}), obtivemos as estimativas das fontes mostradas nas Figuras \ref{fig:icaebm} e \ref{fig:natica} para os métodos ICA-EBM e Natural ICA, respectivamente. Para este caso, ambos os métodos geraram boas estimativas das fontes, além de apresentarem desempenho similar.
    
    Posteriormente, aumentou-se o tempo de reverberação $T_{60}$ do ambiente de simulação, mantendo-se o comprimento da janela $L=2048$. Nas Figuras \ref{fig:icaebmsrc1} e \ref{fig:icaebmsrc2} (para o algoritmo ICA-EBM) e \ref{fig:naticasrc1} \ref{fig:naticasrc2} (para o algoritmo Natural ICA + FastICA), é mostrado como os indicadores de desempenho descritos na Seção \ref{metrics} evoluem com a variação deste parâmetro. As Figuras \ref{fig:icaebmconvergencets} e \ref{fig:naticaconvergencets} mostram o tempo de convergência para cada escolha do tempo de reverberação $T_{60}$ dos algoritmos ICA-EBM e Natural ICA + FastICA, respectivamente. Devido à queda no valor destas métricas com o aumento do tempo de reverberação $T_{60}$, é possível concluir que ambos os algoritmos passaram a ter suas estimativas degradadas. Além disso, apesar do desempenho ligeiramente superior do algoritmo ICA-EBM para todos os tempos de reverberação $T_{60}$, o algoritmo Natural ICA + FastICA apresenta um tempo de convergência significantemente menor para todos os tempos de reverberação $T_{60}$.
    
    Por fim, fixou-se o tempo de reverberação $T_{60}=0.5s$ e variou-se o tamanho da janela $L$ para os valores $256$, $512$, $1024$ e $2048$. O efeito deste parâmetro sobre as métricas de desempenho da Seção \ref{metrics} está representado nas Figuras \ref{fig:icaebmwindowsrc1} e \ref{fig:icaebmwindowsrc2} para o algoritmo ICA-EBM e \ref{fig:naticawindowsrc1} e \ref{fig:naticawindowsrc2} para o algoritmo Natural ICA + FastICA. As Figuras \ref{fig:icaebmconvergencewindow} e \ref{fig:naticaonvergencewindow} mostram o tempo de convergência dos algoritmos ICA-EBM e Natural ICA + FastICA, respectivamente, em função do comprimento da janela $L$. Podemos concluir que, dado este tempo de reverberação, a janela de comprimento $L=512$ apresenta melhor desempenho em ambos os métodos, mas o ICA-EBM leva consideravelmente menos tempo para convergir do que o Natural ICA + FastICA, mostrando-se mais adequado.
    
    \begin{figure}
        \centering
        \includegraphics[scale=0.7]{sources.jpg}
            \caption{Sinais de cada uma das fontes no domínio do tempo.}
        \label{fig:sources}
        \includegraphics[scale=0.7]{sensors.jpg}
            \caption{Sinais em cada um dos sensores no domínio do tempo para $T_{60} = 0.1s$.}
        \label{fig:sensors}
    \end{figure}
    
    \begin{figure}
        \centering
        \includegraphics[scale=0.7]{estimatives_FASTICA.jpg}
        \caption{Sinais de estimativa das fontes obtidas pelo algoritmo ICA-EBM no domínio do tempo para $T_{60} = 0.1s$.}
        \label{fig:icaebm}
        \includegraphics[scale=0.7]{estimatives_NATICA.jpg}
        \caption{Sinais de estimativa das fontes obtidas pelo algoritmo Natural ICA + FastICA  no domínio do tempo para $T_{60} = 0.1s$.}
        \label{fig:natica}
    \end{figure}
    
\begin{figure}
    \centering
    \subfigure[Métricas de desempenho para a estimativa da fonte 1.]
    {
        \includegraphics[scale=0.6]{figuras/reverb_estimative_1_fastica.png}
        \label{fig:icaebmsrc1}
    }
    \\
    \subfigure[Métricas de desempenho para a estimativa da fonte 2.]
    {
        \includegraphics[scale=0.6]{figuras/reverb_estimative_2_fastica.png}
        \label{fig:icaebmsrc2}
    }
    \caption{Métricas de desempenho do algoritmo ICA-EBM em função do tempo de reverberação $T_{60}$. É possível verificar a queda do desempenho através da evolução dos valores das métricas SIR, SDR e SAR provocado pelo aumento do tempo de reverberação.}
    \label{fig:icaebmreverb}
\end{figure}

       
\begin{figure}
    \centering
    \subfigure[Métricas de desempenho para a estimativa da fonte 1.]
    {
        \includegraphics[scale=0.6]{figuras/reverb_estimative_1_natica.png}
        \label{fig:naticasrc1}
    }
    \\
    \subfigure[Métricas de desempenho para a estimativa da fonte 2.]
    {
        \includegraphics[scale=0.6]{figuras/reverb_estimative_2_natica.png}
        \label{fig:naticasrc2}
    }
    \caption{Métricas de desempenho do algoritmo Natural ICA + FastICA em função do tempo de reverberação $T_{60}$. É possível verificar a queda do desempenho através da evolução dos valores das métricas SIR, SDR e SAR provocado pelo aumento do tempo de reverberação.}
    \label{fig:naticareverb}
\end{figure}

       
\begin{figure}
    \centering
    \subfigure[Tempo de convergência para o algoritmo ICA-EBM.]
    {
        \includegraphics[scale=0.5]{figuras/convergence_ts_fastica.png}
        \label{fig:icaebmconvergencets}
    }
    \\
    \subfigure[Tempo de convergência para o algoritmo Natural ICA + FastICA.]
    {
        \includegraphics[scale=0.5]{figuras/convergence_ts_natica.png}
        \label{fig:naticaconvergencets}
    }
    \caption{Tempo de convergência dos algoritmos ICA-EBM e Natural ICA + FastICA em função do tempo de reverberação do ambiente $T_{60}$. É possível verificar que ambos os algoritmos levam mais tempo para convergir conforme o tempo de reverberação aumenta, mas o algoritmo Natural ICA + FastICA leva consideravelmente menos tempo do que o ICA-EBM para qualquer $T_{60}$.}
    \label{fig:naticareverb}
\end{figure}




 
\begin{figure}
    \centering
    \subfigure[Métricas de desempenho para a estimativa da fonte 1.]
    {
        \includegraphics[scale=0.6]{figuras/window_estimative_1_Ts05_fastica.png}
        \label{fig:icaebmwindowsrc1}
    }
    \\
    \subfigure[Métricas de desempenho para a estimativa da fonte 2.]
    {
        \includegraphics[scale=0.6]{figuras/window_estimative_2_Ts05_fastica.png}
        \label{fig:icaebmwindowsrc2}
    }
    \caption{Métricas de desempenho do algoritmo ICA-EBM em função do comprimento da janela $L$. A janela de comprimento $L=512$ possui um desempenho superior.}
    \label{fig:icaebmwindow}
\end{figure}

       
\begin{figure}
    \centering
    \subfigure[Métricas de desempenho para a estimativa da fonte 1.]
    {
        \includegraphics[scale=0.6]{figuras/window_estimative_1_Ts05_natica.png}
        \label{fig:naticawindowsrc1}
    }
    \\
    \subfigure[Métricas de desempenho para a estimativa da fonte 2.]
    {
        \includegraphics[scale=0.6]{figuras/window_estimative_1_Ts05_natica.png}
        \label{fig:naticawindowsrc2}
    }
    \caption{Métricas de desempenho do algoritmo ICA-EBM e Natural ICA + FastICA em função do comprimento da janela $L$. A janela de comprimento $L=512$ possui um desempenho superior.}
    \label{fig:naticawindow}
\end{figure}


\begin{figure}
    \centering
    \subfigure[Tempo de convergência para o algoritmo ICA-EBM.]
    {
        \includegraphics[scale=0.5]{figuras/convergence_fastica.png}
        \label{fig:icaebmconvergencewindow}
    }
    \\
    \subfigure[Tempo de convergência para o algoritmo Natural ICA + FastICA.]
    {
        \includegraphics[scale=0.5]{figuras/convergence_natica.png}
        \label{fig:naticaonvergencewindow}
    }
    \caption{Tempo de convergência dos algoritmos ICA-EBM e Natural ICA + FastICA em função do comprimento da janela $L$. O algoritmo ICA-EBM possui um tempo de convergência mais baixo até a janela de $L=1024$. Após este ponto, o algoritmo Natural ICA + FastICA passa a ter uma queda bem mais acentuada no seu tempo de convergência em relação ao ICA-EBM. Se considerarmos a janela de comprimento $L=512$ como sendo a melhor escolha, o algoritmo ICA-EBM se mostra mais vantajoso. }
    \label{fig:naticawindowconvergence}
\end{figure}

